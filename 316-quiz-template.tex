\newcommand{\tech}[1]{ {\footnotesize [Tech: #1]} }

\newcommand{\solution}[2]{#2}

\newcommand{\ds}{\displaystyle}

\pagestyle{empty}

\newcommand{\quizdetail}{{\huge\noindent{Quiz \QuizNumber} \\{ \normalsize
      {\course --- Due \dueDate %\\
%Covers: \covers
} }\\
\normalsize
}}

\newcommand{\ansbox}{ \rput(-.6in,-.25in){{\begin{tabular}{|c|c|c|}\hline Or &\quad\quad \\\hline C1 &\\ \hline C2 &\\\hline WE &\\ \hline\end{tabular}}}}


\newcommand{\firstpage}{

\noindent \begin{pspicture}(0,0)(0,0)
  \rput(4.5in,.25in){Name:}
\end{pspicture}
\quizdetail

\noindent Take home quiz instructions:

Quizzes are to be done without help from any human being. You are allowed to your brain, a
writing device, prayer, and the one page lesson plan you created yourself (no other notes, text, calculator, etc.). You should not discuss the contents of a quiz with anyone until after the due date. Quizzes are designed to be doable in 20-30 minutes or less, provided ample practice has been completed. You should complete the quiz in one continuous chunk of time (starting the quiz, and then coming back to it an hour later is not permitted). Once you have finished taking your quiz, if you know you have missed some problems then you can immediately begin quiz corrections. Make sure you place any corrections on another sheet of paper.


\bigskip \noindent Quiz correction instructions:

I will grade quizzes and let you know which problems you got right or wrong. You are then responsible for finding your errors, explaining what mistakes you made, and correctly reworking incorrect solutions. You are allowed to use any resource you like to correct your work, though your submitted solutions must be your own work (plagiarism is considered cheating). You may submit quiz corrections at most twice. If you complete your quiz corrections on the first try within one week of the due date on the quiz, I will give you 1/2 of your points back on the original quiz score. Joining a study group which meets regularly will help you accomplish this each week. I provide lots of feedback on quiz corrections after your first attempt.

When submitting corrections, follow these rules:
\begin{enumerate}
 \item Give a written explanation why what you submitted was incorrect. Be specific and explain exactly where and what mistake was made.
 \item Correctly rework each problem for which you did not receive full credit. If you made a minor mistake, you do not need to rework the entire problem, just the portion where your mistake was made.
 \item Submit your original work, unmodified, with your corrections stapled to the back. If you loose your original, you can print a new copy from the web. Submitted corrections should be clearly labeled and in ascending order (problem 1 should appear before problem 2).
\end{enumerate}

A valuable part of the learning process is learning to find your own mistakes and figuring out how to fix them. When you discover that you have a made the same mistake on multiple problem, set a goal for how you plan to avoid making that mistake in the future.


\bigskip \noindent Technology Instructions:

Your final assignment pertaining to each unit requires the use of Technology. Our goal with technology in this class is to become familiar with the syntax of a computer algebra system. On each question you will find a set of Technology instructions, \tech{}, which describe what you should do on this problem with regards to technology. Many of the problems will just say ``repeat,'' which means you should repeat the calculation with the computer. Sometimes the instructions will ask you to do a little more, such as graph a solution. The point to working on the same problems with technology is two-fold. (1) You obtain practice using technology on problems where you already know the answer, so you can focus on learning how to communicate with the computer. (2) Through the technology assignments, you review the entire semester. 

I suggest that you complete each technology assignment prior to its corresponding exam. The due date for technology assignments is the last day of the semester.  Once you complete the technology portion of the quiz, please report that you have done so via I-learn. 

%Here are my suggestions for getting the most out of quiz corrections.
%\begin{enumerate}
%	\item First try to complete your corrections individually. If you cannot tell what your mistake was, then carefully check each of your steps, giving a justification of each step. Often you will discover your own mistakes by doing this. If you were not sure of how to begin, then return to the homework and practice more examples.
%	\item If you have spent more than 5-15 minutes on a problem, and have no clue what is wrong, please get some help. Feel free to ask your peers, a tutor, or me. You may get help from any resource you want.
%	\item When asking others for help, ask if they will look over your work for you. Have your helpers ask you questions about what you did to help you find your errors. Past experience has shown that if you just have helpers tell you what is wrong with your work and how to get the right answer, then you may not learn from your mistakes.
%	\item I strongly suggest that you meet with a study group at least once a week to share what you have learned. This would be a great time to share with each other what you have learned as you corrected your quizzes, as well as a perfect opportunity to teach one another for 25-30 minutes from your lesson plan. 
%\end{enumerate}

\bigskip \noindent The following box appears next to each problem, and is used to record grades.

\begin{tabular}{|c|c|c|}\hline Or &Your original score (out of 2 points) \\\hline C1 & Your score after first correction (if needed)\\ \hline C2 & Your score after second correction (if needed)\\\hline WE & \textbf{W}ritten \textbf{E}xplanation why what you submitted was incorrect\\ \hline\end{tabular}


\newpage
\noindent \begin{pspicture}(0,0)(0,0)
  \rput(5in,.25in){Name:}
  \rput(5in,.1in){\small Estimated time to take:}
  \rput(2.5in,.25in){\begin{tabular}{|c|c|c|c|}\hline Orig. &Cor1 &Bonus& Cor2. \\ \hline &&Correct by &\\\quad/\totalpoints&&\corrdead&\\ \hline\end{tabular}}
\end{pspicture}
\quizdetail

}