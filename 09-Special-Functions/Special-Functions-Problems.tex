

\section{Preparation}

\noindent The name of this module is ``Special Functions.'' These are functions which show up as the solution to problems of real world interest whose solution is a function which does not appear in regular calculus.  Many of these functions can only be represented using series. 
When you make your lesson plan, make sure that you have an example illustrating each idea.


\begin{enumerate}
\item Explain the Frobenius method and use it to solve ODEs where zero is a regular singular point.
\item Solve Legendre's equation and derive Legendre polynomials.
\item Solve Bessel's equation giving solutions of the first kind, and write solutions to ODEs in terms of Bessel functions.
\item Describe the Gamma function and how it generalizes the factorial. Be able to prove various relationships related to the Gamma function.
\end{enumerate}




\begin{center}
\begin{tabular}{ll}
&Preparation Problems
(\href{https://ilearn.byui.edu/bbcswebdav/institution/Physical\_Sci\_Eng/Mathematics/Personal\%20Folders/WoodruffB/316/09-Special-Functions-Preparation-Solutions.pdf}{click for solutions})
\\
\hline\hline
Day 1&28.4, 28.5, 28.6, 28.9\\ \hline
Day 2&27.12, 29.4, 28.10, 28.14\\ \hline
Day 3&30.4, 30.6, 30.8, 30.9\\ \hline
\end{tabular}
\end{center}


Section numbers correspond to problems from Schaum's Outlines \textit{Differential Equations} by Rob Bronson. The suggested problems are a minimum set of problems to attempt. 


\begin{center}
\begin{tabular}{|l|c|l|l|l|l|}
\hline
Concept	&Sec.	&Suggested	&Relevant\\\hline
Frobenius Method*&&&28:1-4, 5-10, 12,14,16,18-20\\ \hline
Legendre Polynomials&&&27:11-13; 29:4,6,8,11,12,15 \\ \hline
Bessel Functions&&&30:9,11,12,26, 27,\\ \hline
Gamma Functions&&& 30:1-8, 24, 25\\ \hline
Substitutions&&&28:22-23, 34-38; 30:30,31 \\ \hline
\end{tabular}
\end{center}

Here are some vocabulary reminders to help you.

\begin{itemize}
	\item 
Remember, a function is said to be analytic at $x=c$ if it has a power series solution centered at $x=c$ with a positive radius of convergence.  Polynomials, exponentials, trig function, and rational functions whose denominator is not zero at $x=c$ are all analytic.
\item
The point $x=c$ is called an ordinary point of the differential equation $y''+P(x) y' +Q(x) y =r(x)$ if both coefficients $P(x)$ and $Q(x)$ are analytic at $x=c$. We can solve ODEs at regular points by using the power series method.
\item The point $x=c$ is called a singular point of the ODE $y''+P(x) y' +Q(x) y =r(x)$ if either coefficient $P(x)$ or $Q(x)$ is not analytic at $x=c$. In general, we can't solve ODEs at singular points.
\item The point $x=c$ is called a regular singular point of the ODE $y''+P(x) y' +Q(x) y =r(x)$  it is singular and if $(x-c)P(x)$ and $(x-c)^2Q(x)$ are both analytic. We can use the Frobenius method to solve ODEs at regular singular points. The big idea is to guess a solution of the form $\ds y = x^\lambda \sum_{n=0}^\infty a_n x^n$ and then solve for $\lambda$ and the remaining coefficients as in the power series method.  The indicial equation is the first equation resulting from matching coefficients, and it's roots $\lambda_1$ and $\lambda_2$ determine the type of solution.
\end{itemize}

