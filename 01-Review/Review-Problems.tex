\section{Preparation}

\noindent 
This chapter covers the following ideas. When you create your lesson plan, it should contain examples which illustrate these key ideas. Before you take the quiz on this unit, meet with another student out of class and teach each other from the examples on your lesson plan. 


\begin{enumerate}

\item Graph basic functions by hand. Compute derivatives and integrals, in particular using the product rule, quotient rule, chain rule, integration by $u$-substitution, and integration by parts (the tabular method is useful for simplifying notation). Explain how to find a Laplace transform. 
\item Explain how to verify a function is a solution to an ODE, and illustrate how to solve separable ODEs.
\item Explain how to use the language of functions in high dimensions and how to compute derivatives using a matrix. Illustrate the chain rule in high dimensions with matrix multiplication.
\item Graph the gradient of a function together with several level curves to illustrate that the gradient is normal to level curves.
\item Explain how to test if a differential form is exact (a vector field is conservative) and how to find a potential. 

\end{enumerate}





Most days of class we will present in groups some material you have prepared, or work problems similar to the preparation problems below. Typically there will be 4 problems for each day. Each member of the group should prepare one of these problems and then in class you will have the opportunity to teach your group what you have learned as you work through problems. You will occasionally select a problem which is entirely new to you, which you have never seen modeled before. If this occurs, you should look for examples similar to this problem in the text, and follow those examples to learn how to do this problem. You will be exercising your faith to then go and teach the class something you have never before seen modeled, and your confidence will grow. These problems will normally be the 4th one listed on the preparation problems, so I suggest that as a group you alternate who takes this problem so that you all get a chance to grow. As time permits, I will post hand written solutions to these problems on the course website.  


Here are the preparation problems for this unit.  Problems that come from Schaum's Outlines are preceded by a chapter number. The problems that start with an H are located just below this list.   Realize that sometimes the method of solving the problem in Schaum's Outlines will differ from how we solve the problem in class. 




\begin{center}
\begin{tabular}{ll|l}
\multicolumn{2}{c}{Preparation Problems (\href{https://ilearn.byui.edu/bbcswebdav/institution/Physical\_Sci\_Eng/Mathematics/Personal\%20Folders/WoodruffB/316/01-Review-Preparation-Solutions.pdf}{click for handwritten solutions})}
%&
%Webcasts 
%(
%\href{http://ilearn.byui.edu/bbcswebdav/institution/Physical\_Sci\_Eng/Mathematics/Personal\%20Folders/WoodruffB/341/3-Patterns-videos.pdf}{pdf copy}
%)
\\
\hline\hline
Day 1&
21.5, 
21.29, 
4.2, 
4.8
%&
%\href{http://ilearn.byui.edu/bbcswebdav/institution/Physical\_Sci\_Eng/Mathematics/Personal\%20Folders/WoodruffB/341/3-Patterns-video-01.wmv}{1},
%\href{http://ilearn.byui.edu/bbcswebdav/institution/Physical\_Sci\_Eng/Mathematics/Personal\%20Folders/WoodruffB/341/3-Patterns-video-02.wmv}{2},
%\href{http://ilearn.byui.edu/bbcswebdav/institution/Physical\_Sci\_Eng/Mathematics/Personal\%20Folders/WoodruffB/341/3-Patterns-video-03.wmv}{3}
\\ \hline
Day 2&
1.2, 
1.7, 
H.1c, 
H.2b
%&
%\href{http://ilearn.byui.edu/bbcswebdav/institution/Physical\_Sci\_Eng/Mathematics/Personal\%20Folders/WoodruffB/341/3-Patterns-video-04.wmv}{4},
%\href{http://ilearn.byui.edu/bbcswebdav/institution/Physical\_Sci\_Eng/Mathematics/Personal\%20Folders/WoodruffB/341/3-Patterns-video-05.wmv}{5},
%\href{http://ilearn.byui.edu/bbcswebdav/institution/Physical\_Sci\_Eng/Mathematics/Personal\%20Folders/WoodruffB/341/3-Patterns-video-06.wmv}{6}
\\ \hline
Day 3&
H.3c,
H.4a, 
H.5b, 
H.4c
%&
%\href{http://ilearn.byui.edu/bbcswebdav/institution/Physical\_Sci\_Eng/Mathematics/Personal\%20Folders/WoodruffB/341/3-Patterns-video-07.wmv}{7},
%\href{http://ilearn.byui.edu/bbcswebdav/institution/Physical\_Sci\_Eng/Mathematics/Personal\%20Folders/WoodruffB/341/3-Patterns-video-08.wmv}{8},
%\href{http://ilearn.byui.edu/bbcswebdav/institution/Physical\_Sci\_Eng/Mathematics/Personal\%20Folders/WoodruffB/341/3-Patterns-video-09.wmv}{9}
\\ \hline
\end{tabular}
\end{center}


Aside from learning the terms ``Laplace Transforms'' and ``Separable ODEs'' (both of which only require that you practice your integration), this unit contains material which is review material.  The amount and type of reviewing that we each needs to do will be unique.   Remember that your homework assignment is to do enough of each type of problem to master the material we are learning (with a minimum of 7 problems per day of class). Pick problems that will help you develop your skills.
The following problems relate to what we are studying in class.
Section numbers correspond to problems from Schaum's Outlines \textit{Differential Equations} by Rob Bronson. The suggested problems are a minimum set of problems to attempt. 

\begin{center}
\begin{tabular}{|l|c|l|l|l|l|}
\hline
Concept&Sec.&Suggestions&Relevant Problems\\ \hline
Laplace Transforms&21&5,29,38,63&1-6,10,13,27-32,36-38,43,,45,47,48,50,53,63-65\\ \hline
Vocabulary of ODEs&1&2,6,7,26,29,40&1-13,14-54\\ \hline
Separable ODEs&4&2,5,8,26,34,40,43&1-8,23-45 (get lots of practice with integration)\\ \hline
Derivatives and the Chain Rule&here&1ace,2abd&1,2\\ \hline
Gradients and Level Curves&here&3ace&3\\ \hline
Finding potentials&here&4ac, 5bf&4,5\\ \hline
\end{tabular}
\end{center}















\section{Problems}
\begin{enumerate}
	\item For each function, graph the function and find its derivative (as a matrix). This reviews the types of functions you encountered in multivariable calculus. (The Mathematica code online has examples of all of these).
%\begin{multicols}{2}
\begin{enumerate}
	\item $\vec r(t)=\left<3\cos t, 2 \sin t\right>, 0\leq t\leq 3\pi/2$
	\item $\vec r(t)=\left<4\cos t, 3\sin t,2t\right>, 0\leq t\leq 2\pi$
	\item $f(x,y)=4-x^2-y^2, x^2+y^2\leq 4$ (draw both the surface and several level curves)
	\item $f(x,y,z)=x^2+y^2+z^2$ (draw several level surfaces)
	\item $\vec F(x,y)=\left<-y, x\right>$ for $-3\leq x\leq 3$ $-3\leq y\leq 3$
	\item $\vec F(x,y,z)=\left<-x, -y,-z\right>$ for $-2\leq x\leq 2,-2\leq y\leq 2,-2\leq z\leq 2$
	\item $\vec r(u,v)=\left<u\cos v, u\sin v, u \right>$ for $0\leq u\leq 1, 0\leq v\leq 2\pi$.
\end{enumerate}
%\end{multicols}

	\item Use the chain rule with matrix multiplication to find the following derivatives.
%\begin{multicols}{2}
\begin{enumerate}
	\item Find $\frac{df}{dt}$ if $f(x,y)=x^2y$ and $x=\cos t, y=\sin t$.
	\item Find $f_u$ and $f_v$ if $f(x,y)=3x-4y$ and $x=2u-v, y=6uv$.
	\item Find $f_r$ and $f_\theta$ if $f(x,y)=9-x^2-y^2$ and $x=r\cos \theta, y=r\sin\theta$.
	\item Find $\vec r_u$ and $\vec r_v$ if $\vec r(x,y)=\left<x,y,x^2-y\right>$ and $x=2u-v, y=6uv$.
	\item Find $\vec F_r$ and $\vec F_\theta$ if $\vec F(x,y)=\left<-y,x\right>$ and $x=r\cos\theta, y=r\sin\theta$.
	\item Find $f_r, f_\theta,$ and $f_z$ if $\vec f(x,y,z)=x^2+4yz$ and $x=r\cos\theta, y=r\sin\theta, z=z$ (cylindrical coordinates).
\end{enumerate}
%\end{multicols}

	\item For each of the following functions, construct a graph which contains both the gradient and several level curves (Try using the code in Mathematica to help you check your work).
\begin{enumerate}
\begin{multicols}{2}
	\item $f(x,y)=x+2y$
	\item $f(x,y)=-x+2y$
	\item $f(x,y)=x^2+y$
	\item $f(x,y)=x^2-y$
	\item $f(x,y)=x+y^2$
	\item $f(x,y)=x-y^2$
\end{multicols}
\end{enumerate}

	\item For each of the following vector fields, use the test for a conservative vector field to determine if the vector field has a potential.  If it has a potential, then find a potential.
\begin{enumerate}
\begin{multicols}{2}
	\item $\vec F(x,y) = \left<4x+5y,5x+6y\right>$
	\item $\vec F(x,y) = \left<2x-y,x+2y\right>$
	\item $\vec F(x,y) = \left<e^{3x}+e^{2y},2xe^{2y}-\frac{1}{1+y^2}\right>$
	\item $\vec F(x,y) = \left<4x+5y,5x+6y\right>$
	\item $\vec F(x,y,z) = \left<x+y+z,x+y+z, x+y+z\right>$
	\item $\vec F(x,y,z) = \left<3y+yz, 3x+xz+2y+5z, zy+5y\right>$
\end{multicols}
\end{enumerate}
	
	
	\item For each of the following differential forms, test to see if the differential form is exact.  If it is exact, find a function whose differential is the differential form.
\begin{enumerate}
\begin{multicols}{2}
	\item $(4x+5y)dx+(5x+6y)dy$
	\item $(2x-y)dx+(x+2y)dy$
	\item $(e^{3x}+e^{2y})dx+(2xe^{2y}-\frac{1}{1+y^2})dy$
	\item $(4x+5y)dx+(5x+6y)dy$
\end{multicols}
	\item $(x+y+z)dx+(x+y+z)dy+( x+y+z)dz$
	\item $(x+3y+yz)dx+(3x+xz+2y+5z)dy+(zy+5y+4z)dz$
\end{enumerate}
	
\end{enumerate}

\section{Solutions}
The solutions to problems from Schaum's Outlines are self contained.  Hand written solutions to all these problems are available online.  \href{https://ilearn.byui.edu/bbcswebdav/institution/Physical\_Sci\_Eng/Mathematics/Personal\%20Folders/WoodruffB/316/01-Review-Preparation-Solutions.pdf}{Click for the solutions}
